% \documentclass[]{llncs}
\documentclass[a4paper,12pt]{article}

% \def \lncs {}

%BOF %Refer to Chronicle.tex for the changes introduced in the Macros.tex file.
% \input{../../../Common/Chronicle.tex}
% \input{Chronicle.tex}

%To do: 
% Accents causing problems: \v \c

%BEGIN
%COMMAND STRUCTURE: capital letters are types and small letters are modifiers 
%\eg \VX denotes a random variable X, whereas \v\VX denotes a vector of random variables \VX[1]...\VX[n]
% A: algorithm; a: adversary
% B b
% C: complexity class c: circuit
% D d 
% E: Event; e: 
% F f: function 
% G: Graph g: group?
% H h 
% I i 
% J j 
% K k 
% L: Language; l 
% M: matrix; m:
% N n 
% % O o: operator
% P p: polynomial
% Q q
% R: reference; r 
% S: sample space/distribution/set; s: simulator 
% T: Text type; t:math text type
% U u 
% V: Vector v: random variable
% W w 
% X x 
% Y y 
% Z z
% Supscripts/superscripts
% @ is the superscript/subscript operator
%\A: adversary 
%\
% 
% 
% 
% 
%END

%%%%%%%%%%%%%%%%%%%%%%%%%%%%%%%%%%%%%%%%%%%%%%%%%%%%%%%%%%%%%%%%%%%%%%%%%%%%%%%%%%%%%%%%%%%%%%%%%%%%%%%%%%%
%PACKAGES
%BEGIN
%Base packages
\usepackage[utf8]{inputenc}
\usepackage[normalem]{ulem}
% \usepackage{at}

%Font packages
\usepackage{amssymb,amsmath,wasysym,dsfont,bm,relsize,amsbsy}%ntheorem
% \usepackage{lmodern,charter,mathptmx}
\usepackage{mathtools,bigints}%mathrsfs
% \usepackage[cal=euler,calscaled=.96]{mathalfa}
\usepackage{fancyvrb}
\usepackage{authblk}
\usepackage{enumerate}
\usepackage{xparse}
\ifx \slide \undefined
  \usepackage[usenames,dvipsnames,svgnames,table]{xcolor}
  \usepackage{paralist}
\else
%   \usepackage[orientation=portrait, size=a0, scale=1.4]{beamerposter}
%   \usepackage{texpos}
\fi
\usepackage[boxed]{algorithm}
\usepackage{algpseudocode,algorithmicx}
\usepackage{boxedminipage}
\usepackage{etoolbox}
\usepackage{xspace,xargs,xstring}
% \usepackage{nomencl}  
\usepackage{color,framed,mdframed}
\usepackage{caption,subcaption}
\usepackage{appendix,booktabs,array,multirow}
\usepackage{pgffor}
\usepackage{graphicx}
\usepackage{marvosym}
\usepackage{pdfpages}
\usepackage[export]{adjustbox}


\usepackage{pgfplots}
% \useoutertheme{infolines}
% \usepackage{footmisc}

% \usepackage{memoir}
% \usepackage{breqn}

%Graphics packages
\usepackage{tikz}
\usetikzlibrary{arrows,chains,matrix,positioning,scopes,decorations.markings,patterns,shapes.callouts,snakes,shapes.multipart,decorations.pathreplacing,calc,shapes.geometric,backgrouds}
% \DeclareMathAlphabet{\mathpzc}{OT1}{pzc}{m}{it}

%Theorems/definitions
\ifx \lncs \undefined
  \usepackage{amsthm}
%   \let\proof\relax 
  \newtheorem{theorem}{Theorem}
  \newtheorem*{theorem*}{Theorem}
  \newtheorem{lemma}{Lemma}
  \newtheorem{Xlemma}{Lemma}[theorem]
  \newtheorem*{result}{Main Result}
  \newtheorem{corollary}{Corollary}
  \newtheorem*{corollary*}{Corollary}
  \newtheorem{observation}{Observation}
  \newtheorem{claim}{Claim}[theorem]
  \newtheorem{conjecture}{Conjecture}

  %Inline theorems
  \makeatletter
  \newenvironment{ilemma}[1][]
    {\enskip\refstepcounter{lemma}\textbf{Lemma~\thelemma} \ifthenelse{\equal{#1}{}}{}{~(#1)}.\space\itshape}% \begin{shortthm}
  \makeatother
  
  \theoremstyle{definition}
  \newtheorem{definition}{Definition}
  \newtheorem{remark}{Remark}
%   \newtheorem{Comment}{Comment}
  \newtheorem{assumption}{Assumption}
  \newtheorem{example}{Example}
  
  %Inline definitions
  \makeatletter
  \newenvironment{idefinition}[1][]
    {\enskip\refstepcounter{definition}\textbf{Definition~\thedefinition} \ifthenelse{\equal{#1}{}}{}{~(#1)}.\space\itshape}% \begin{shortthm}
  \makeatother
  
%   \let\endproof\relax
  
%   \theoremstyle{nonumberplain}
%   \newtheorem{handnum-theorem}{Theorem}
\fi

\let\oldparagraph\paragraph   %% otherwise paragraph headings are drowned by all the \bf stuff around them
\renewcommand{\paragraph}[1]{\oldparagraph{\bf #1}}
% \let\oldundottedi\i
% \renewcommand{\i}{\textit{\oldundottedi}}

\newfloat{construction}{t!}{lop}
\floatname{construction}{Construction}
\newfloat{game}{t!}{lop}
\floatname{game}{Game}

% \newtheorem*{theorem*}{Theorem}

%%%%%%%%%%%%%%%%%%%%%%%%%%%%%%%%%%%%%%%%%%%%%%%%%%%%%%%%%%%%%%%%%%%%%%%%%%%%%%%%%%%%%%%%%%%%%%%%%%%%%%%%%%
%END

%To-do: move to xparse

%%%%%%%%%%%%%%%%%%%%%%%%%%%%%%%%%%%%%%%%%%%%%%%%%%%%%%%%%%%%%%%%%%%%%%%%%%%%%%%%%%%%%%%%%%%%%%%%%%%%%%%%%%%
%TEXT-TYPES \T, \t
%BEGIN
\newcommandx{\TA}[1]{\textrm{\sf #1}} %Algorithm
\newcommandx{\TB}[1]{\textrm{\bf #1}} %Bold 
\newcommandx{\TK}[1]{\textrm{\tt #1}} %Keys, events, procedures/algorithms
\newcommandx{\TE}[1]{\textit{#1}} %Emphasised 
\newcommandx{\TC}[1]{\text{\smaller #1}} %Characteristic
\newcommandx{\TP}[1]{#1} %Plain
\newcommandx{\TS}[1]{\text{#1}} %Straight
\newcommandx{\TEB}[1]{{\textbf{\textit #1}}} %Bold and italic: for theorems
\newcommandx{\TR}[1]{#1} %Text used for references

%Math text-types
\newcommandx{\tA}[1]{\mathsf{#1}} %Algorithm
\newcommandx{\tB}[1]{\boldsymbol{#1}} %Highlighted 
\newcommandx{\tK}[1]{\mathtt{#1}} %Keys
\newcommandx{\tE}[1]{\mathit{#1}} %Emphasised
\newcommandx{\tC}[1]{\mathsf{#1}} %Characteristic
\newcommandx{\tO}[1]{\mathds{#1}} %Operator
\newcommandx{\tF}[1]{#1} %Function
\newcommandx{\tS}[1]{\mathcal{#1}} %sample space
%END

%%%%%%%%%%%%%%%%%%%%%%%%%%%%%%%%%%%%%%%%%%%%%%%%%%%%%%%%%%%%%%%%%%%%%%%%%%%%%%%%%%%%%%%%%%%%%%%%%%%%%%%%%%
%CORE COMMANDS: series of commands \AA...\AZ generated later.
%BEGIN
\newcommandx{\Algorithm}[1]{\TA{#1}\xspace} %Algorithm \A
\newcommandx{\circuit}[1]{\TA{#1}\xspace} %Circuit \c
\newcommandx{\set}[1]{\tS{#1}\xspace} %Set \s
\newcommandx{\Distribution}[1]{\tS{#1}\xspace} %Sample space/Distribution \D
\newcommandx{\Class}[1]{\TA{#1}\xspace} %Complexity Class \C
\newcommandx{\Language}[1]{\TC{#1}\xspace} %Language \L
\newcommandx{\Graph}[1]{\tF{#1}\xspace} %Graph $\G
\newcommandx{\function}[1]{\tF{#1}\xspace} %Function \f
\newcommandx{\polynomial}[1]{\tF{#1}\xspace} %polynomial \p 
\newcommandx{\variable}[1]{\tF{#1}\xspace} %Random variable \v
\newcommandx{\Event}[1]{\tF{#1}\xspace} %Event \E
\newcommandx{\Matrix}[1]{\tB{#1}\xspace} %Matrix \M
\newcommandx{\Vector}[1]{\tB{#1}\xspace} %Vector \V

%Declare batches: automate later
\newcommandx{\DeclareAlgorithm}[1]{%Shorthands
  \expandafter\def\csname A#1\endcsname{\Algorithm{#1}} %Algorithm
}
\newcommandx{\DeclareAlgorithms}{\forcsvlist{\DeclareAlgorithm}}

\newcommandx{\Declarecircuit}[1]{%Shorthands
  \expandafter\def\csname c#1\endcsname{\circuit{#1}} %circuit
}
\newcommandx{\Declarecircuits}{\forcsvlist{\Declarecircuit}}

\newcommandx{\Declareset}[1]{%Shorthands
  \expandafter\def\csname s#1\endcsname{\set{#1}} %Set
}
\newcommandx{\Declaresets}{\forcsvlist{\Declareset}}

\newcommandx{\DeclareDistribution}[1]{%Shorthands
  \expandafter\def\csname D#1\endcsname{\Distribution{#1}} %Distribution
}
\newcommandx{\DeclareDistributions}{\forcsvlist{\DeclareDistribution}}

\newcommandx{\DeclareClass}[1]{%Shorthands
  \expandafter\def\csname C#1\endcsname{\Class{#1}} %Class
}
\newcommandx{\DeclareClasses}{\forcsvlist{\DeclareClass}}

\newcommandx{\DeclareLanguage}[1]{%Shorthands
  \expandafter\def\csname L#1\endcsname{\Language{#1}} %Language
}
\newcommandx{\DeclareLanguages}{\forcsvlist{\DeclareLanguage}}

\newcommandx{\DeclareGraph}[1]{%Shorthands
  \expandafter\def\csname G#1\endcsname{\Graph{#1}} %Graph
}
\newcommandx{\DeclareGraphs}{\forcsvlist{\DeclareGraph}}

\newcommandx{\Declarefunction}[1]{%Shorthands
  \expandafter\def\csname f#1\endcsname{\function{#1}} %function
}
\newcommandx{\Declarefunctions}{\forcsvlist{\Declarefunction}}

\newcommandx{\Declarepolynomial}[1]{%Shorthands
  \expandafter\def\csname p#1\endcsname{\polynomial{#1}} %polynomial
}
\newcommandx{\Declarepolynomials}{\forcsvlist{\Declarepolynomial}}

\newcommandx{\Declarevariable}[1]{%Shorthands
  \expandafter\def\csname v#1\endcsname{\variable{#1}} %variable
}
\newcommandx{\Declarevariables}{\forcsvlist{\Declarevariable}}

\newcommandx{\DeclareEvent}[1]{%Shorthands
  \expandafter\def\csname E#1\endcsname{\Event{#1}} %Event
}
\newcommandx{\DeclareEvents}{\forcsvlist{\DeclareEvent}}

\newcommandx{\DeclareMatrix}[1]{%Shorthands
  \expandafter\def\csname M#1\endcsname{\Matrix{#1}} %Matrix
}
\newcommandx{\DeclareMatrices}{\forcsvlist{\DeclareMatrix}}

\newcommandx{\DeclareVector}[1]{%Shorthands
  \expandafter\def\csname V#1\endcsname{\Vector{#1}} %Vector
}
\newcommandx{\DeclareVectors}{\forcsvlist{\DeclareVector}}

\newcommandx{\DeclareModel}[1]{%Shorthands
  \expandafter\def\csname M#1\endcsname{\Model{#1}} %Matrix
}
\newcommandx{\DeclareModels}{\forcsvlist{\DeclareModel}}


%Batch
\DeclareAlgorithms{K,G,D,M,A,C,B,R,I,F,S,Z,V,X,U,P,R,O,L,iO,H,W,w,PRG,PRF}
\Declarecircuits{C,S,H,h,P,V,T,A}
\DeclareGraphs{G,S,C,K,P,T,H,SC}
\DeclareMatrices{A,K,B,C,M,H,G}
\DeclareVectors{I,k,r,x,y,e,d,a,b,c,w,v,u,q,z,X,A,B,s,P}
\DeclareDistributions{X,V,H,Y,Z,R,K,F,W,S,P,M,C,D,G,A,U}
\Declaresets{R,S,Z,N,A,B,T,X,Y,V,C,H,F,G,E,P,I,K,M,L,D,U}
\DeclareEvents{B,C,D,E,F,G,S,V,W,A}
\Declarevariables{F,T,W,X,I,A,B,G,V,Y,U,H,Z}
\Declarefunctions{f,m,a,r,p,b,g,h,c,s,e,t,w}
\Declarepolynomials{p,q,t}
\DeclareClasses{P,NP,AM,SZK,IP,AM,MA,RP,ZPP,PCP,BPP,ZK,CZK,SZK,PZK,NC,AC,L,NL,TIME,PTIME,NTIME,EXPTIME,NEXPTIME,SPACE,NSPACE,PSPACE,AT,ST,S,CS,T}
\DeclareLanguages{L,A,B,GNI,GI,SAT,CSAT}
\Declarefunctions{RSA,LSB,Rabin,DLP,DES,Auto,Iso,Rev,GGM}
% \DeclareModels{}


%Modifiers: suffixes/prefixes
\newcommandx{\script}[2][1=,2=]{_{#1}^{#2}} %subscript/superscript, later transformed to @
\renewcommandx{\a}{\ddot} %Adversary
\renewcommandx{\S}{\bar} %Simulator
\renewcommandx{\t}{\hat} %target
\newcommandx{\M}{\tilde} %target
\newcommandx{\C}{\bar} %set compliment/bit compliment
\newcommandx{\N}{\neg} %logical negation
% \newcommandx{\overbar}[1]{\mkern 1.5mu\overline{\mkern-1.5mu#1\mkern-1.5mu}\mkern 1.5mu}
\newcommandx{\I}{\script[][-1]} %inverse
\newcommandx{\T}{\script[][T]} %transpose of a matrix
\newcommandx{\s}{\script[][*]} %starred
\newcommandx{\p}{\script[][']} %prime
\renewcommandx{\P}{\script[][+]} %Plus
\newcommandx{\V}{\tB} %vector
\newcommandx{\pars}[1][1=1]{%parameters
  \ifthenelse{\equal{#1}{1}}
    {(\cdot)} %(.)
    {(\cdot\foreach\n in {2,...,#1}{,\cdot})} %(...)
}
\newcommandx{\parS}[1][1=1]{%parameters
  \ifthenelse{\equal{#1}{1}}
    {[\cdot]} %.)
    {[\cdot\foreach\n in {2,...,#1}{,\cdot}} %(...)
}

%Misc
\newcommandx{\fl}[2][1=,2=]{\function[#1][#2]{\ell}} %length function
\newcommandx{\pl}[2][1=,2=]{\polynomial[#1][#2]{\ell}} %polynomial l(.)
\newcommandx{\LTSAT}{\L{3SAT}}
\newcommandx{\LGTC}{\L{G3C}}

%END

%%%%%%%%%%%%%%%%%%%%%%%%%%%%%%%%%%%%%%%%%%%%%%%%%%%%%%%%%%%%%%%%%%%%%%%%%%%%%%%%%%%%%%%%%%%%%%%%%%%%%%%%%%
%MATH/CS COMMANDS
%BEGIN
\newcommandx{\seq}[3][1={,}]{#2#1\ldots#1#3} %A sequence of form a,...,b or in general a op ... op b
\DeclarePairedDelimiter{\Set}{\{}{\}} %Of the form {a}
\newcommandx{\Seq}[2]{\Set{\seq{#1}{#2}}} %A sequence of the form {a,...,b}
\newcommandx{\str}[1][1=*]{\Set{0,1}\script[][#1]} %String

%Logicical
\newcommandx{\AND}{\TC{AND}\xspace}
\newcommandx{\NAND}{\TC{NAND}\xspace}
\newcommandx{\OR}{\TC{OR}\xspace}
\newcommandx{\NOT}{\TC{NOT}\xspace}
\newcommandx{\XOR}{\TC{XOR}\xspace}

\newcommandx{\YES}{\TC{YES}\xspace}
\newcommandx{\NO}{\TC{NO}\xspace}
\newcommandx{\ACCEPT}{\TC{ACCEPT}\xspace} %Turing machine: accept
\newcommandx{\REJECT}{\TC{REJECT}\xspace} %Turing machine: reject

\newcommandx{\HALT}{\TC{HALT}\xspace}
\newcommandx{\state}{\TA{state}\xspace}
\newcommandx{\TRUE}{\TC{TRUE}\xspace}
\newcommandx{\FALSE}{\TC{FALSE}\xspace}

\newcommandx{\READ}[2][1=,2=]{
  \ifthenelse{\equal{#1}{}}
  {\TC{READ}\xspace}
  {(#1~\TC{READ}~#2)}
}
\newcommandx{\OUTPUT}[2][1=,2=]{
  \ifthenelse{\equal{#1}{}}
  {\TC{OUTPUT}\xspace}
  {(#1~\TC{OUTPUT}~#2)}
}
\newcommandx{\OPERATE}[2][1=,2=]{
  \ifthenelse{\equal{#1}{}}
  {\TC{OP}\xspace}
  {(#1~\TC{OP}~#2)}
}

%Relations
\newcommandx{\red}[1][]{\leq_{#1}} %Reduced to <
\newcommandx{\pred}{\red[\TP{P}]} %Polynomial time reduction 
\newcommandx{\tred}{\red[\TP{T}]} %Turing reduction
\newcommandx{\sep}[1][1=]{\nleq_{#1}} %Separation
\newcommandx{\psep}{\sep[\TP{P}]} %Polynomial separation
\newcommandx{\imp}[1][]{\xRightarrow{#1}} %Implies ->
\newcommandx{\pmi}[1][]{\xLeftarrow{#1}} %Implies <-
% \newcommandx{\sepi}{\nRightarrow} %Separation
\newcommandx{\bij}[2][1=,2=]{\xleftrightarrow[#2]{~#1~}} %Bijective
\newcommandx{\und}{\textit{?}} %Undecided
\newcommandx{\cat}{\circ} %String concatenation
\newcommandx{\cont}{\Lightning} %Contradiction
\newcommandx{\tends}{\rightarrow} %Limit
\newcommandx{\forsome}{\TP{ for some }} %For some
\newcommandx{\suchthat}{\TP{ s.t. }} %Such that

%Operators
\DeclarePairedDelimiter{\card}{\lvert}{\rvert} %Cardinality of a set |.|
\DeclarePairedDelimiter{\abs}{\lvert}{\rvert} %Absolute value |.|
\DeclarePairedDelimiter{\size}{\lvert}{\rvert} %size/length of a string |.|
\DeclarePairedDelimiter{\norm}{\lvert\lvert}{\rvert\rvert} %Norm
\DeclarePairedDelimiter{\ceil}{\lceil}{\rceil}
\DeclarePairedDelimiter{\floor}{\lfloor}{\rfloor}
\newcommandx{\JL}[2]{\left(\dfrac{#1}{#2}\right)} %The Legendre/Jacobi symbol
\newcommandx{\Trace}[1]{\TS{Trace}(#1)} %Trace of a linear function
\newcommandx{\Support}[1]{\TS{Support}(#1)} %Support of a probability distribution
\newcommandx{\CPr}[1]{\fp[c](#1)} %Collision probability
\newcommandx{\power}[1]{\mathfrak{P}(#1)} %Power set
\newcommandx{\oand}{\wedge} %Logical and
% \newcommandx{\oAND}{\big\wedge} %Logical and
\newcommandx{\oor}{\vee} %Logical or
% \newcommandx{\oOR}{\big\vee} %Logical or
\newcommandx{\oxor}{\oplus}
% \newcommandx{\oXOR}{\big\oplus}
\DeclarePairedDelimiter{\Tuple}{\langle}{\rangle} %Tuple
\DeclarePairedDelimiter{\Encoding}{\langle}{\rangle} %encoding, say for a Turing machine
\DeclarePairedDelimiter{\IP}{\langle}{\rangle} %Inner product, interactive protocol

%Functions

\RenewDocumentCommand\Pr{sO{}m}{%Probability
  \IfBooleanTF#1%
    {\underset{#2}{\tO{P}}\left[#3\right]}%with braces
    {\underset{#2}{\tO{P}}[#3]}
}
\NewDocumentCommand\Ex{sO{}m}{%Expectation
  \IfBooleanTF#1%
    {\underset{#2}{\tO{E}}\left[#3\right]} %with braces
    {\underset{#2}{\tO{E}}[#3]} 
}
\NewDocumentCommand\Var{sO{}m}{%Variance
  \IfBooleanTF#1%
    {\underset{#2}{\tO{V}}\left[#3\right]} %with braces
    {\underset{#2}{\tO{V}}[#3]} 
}
% \RenewDocumentCommand\min{sO{}m}{%Minimum
%   \IfBooleanTF#1%
%     {\underset{#2}{\TS{min}}\left[#3\right]} %with braces
%     {\underset{#2}{\TS{min}}[#3]} 
% }
% \RenewDocumentCommand\max{sO{}m}{%Maximum
%   \IfBooleanTF#1%
%     {\underset{#2}{\TS{max}}\left[#3\right]} %with braces
%     {\underset{#2}{\TS{max}}[#3]} 
% }
% \RenewDocumentCommand\log{sm}{%Logarithm
%   \IfBooleanTF#1%
%     {\TS{log}(#2)} %with braces
%     {\TS{log}\ #2}
% }
\newcommandx{\dif}[1][1=]{\tF{d} #1\xspace} % d
\newcommandx{\dx}{\dif{x}} % dx
\newcommandx{\dy}{\dif{y}} % dy
\newcommandx{\dt}{\dif{t}} % dt
\newcommandx{\from}[1]{\left.#1\right} %definite integral limit

%Math entries
\newcommandx{\IndC}{\stackrel{c}{\equiv}} %computationally indistinguishable
\newcommandx{\IndS}{\stackrel{s}{\equiv}} %statistically indistinguishable
\newcommandx{\Ind}{\equiv} %Identically distributed
% \newcommandx{\DI}{\sim} %Identically distributed
\newcommandx{\CD}[1][1=c]{\not\stackrel{#1}{\equiv}} %computationally distinguishable
\newcommandx{\DIA}[1][1=]{\approx_{#1}} %Computationally identically distributed, with an error
\newcommandx{\nDIA}[1][1=]{\not\approx_{#1}} %Computationally identically distributed, with an error
\newcommandx{\Equiv}{\Longleftrightarrow} %equivalent to
\newcommandx{\out}{\leftarrow} %Output <-
\newcommandx{\tuo}{\rightarrow} %Output ->
\newcommandx{\outr}{\xleftarrow{\scriptscriptstyle \$}} %Output of a randomised algorithm
\newcommandx{\tuor}{\xrightarrow{\scriptscriptstyle \$}} %Output/Right of a randomised algorithm
\newcommandx{\UR}{\xleftarrow{\scriptscriptstyle \textnormal{U}}} %Uniformly at random
\newcommandx{\BR}[1]{\xleftarrow{\scriptscriptstyle #1}} %random with a bias #1
\newcommandx{\Sle}{\xleftarrow{\scriptscriptstyle \$}} %Sample 
\newcommandx{\Sample}{\sim} %Sample  
\newcommandx{\as}{\bot} %abort symbol
\newcommandx{\es}{\lambda} %empty string
\renewcommand{\sp}{\lambda} %Security parameter
\newcommandx{\usp}{1^\sp} %Unary SP
\newcommandx{\eps}{\epsilon} %Adversarial advantage
\newcommandx{\cross}{\times} %AxBxC
\newcommandx{\op}{\TA{op}\xspace} %Output 
\newcommandx{\opb}{\TA{b}\xspace} %Output bit
\newcommandx{\oList}{\TA{list}\xspace} %Output list
\newcommandx{\union}{\cup}
\newcommandx{\inter}{\cap}
\newcommandx{\Union}{\bigcup}
\newcommandx{\Inter}{\bigcap}
\newcommandx{\cngr}{\equiv} %congruence
\newcommandx{\ifif}[1][1=]{\stackrel{#1}{\Longleftrightarrow}}
\newcommandx{\defeq}{\stackrel{\TP{\tiny def}}{=}\xspace} %defined as
\newcommandx{\iseq}{\stackrel{?}{=}\xspace} %is equal?
\newcommandx{\tie}{\leq\script[\Delta]} %is equal?
\newcommandx{\aseq}{\stackrel{*}{=}\xspace} %asterisk equal -- for marking equations
% \newcommandx{\ST}{\mid} %Such that
\newcommandx{\given}{|} %Given (conditional probability)
\newcommandx{\ON}{:} %Conditioned on (probability)
\newcommandx{\sends}{\longrightarrow}
\newcommandx{\half}{\frac{1}{2}}
\newcommandx{\third}{\frac{1}{3}}
\newcommandx{\twothird}{\frac{2}{3}}
\newcommandx{\fourth}{\frac{1}{4}}
\newcommandx{\eighth}{\frac{1}{8}}

%Remainder
\newcommandx{\fOT}{\f{OT}} %oblivious transfer
\newcommandx{\APV}[2]{(\AP,\AV)} %Prover
% \newcommandx{\ch}[2]{\c{h}} %Circuit h
% \newcommandx{\GSC}[2]{\G{SC}} %Superconcentrator
\newcommandx{\vpi}[2]{\v{\pi}} %vector
% \newcommandx{\vX}[2]{\v{X}} %pROM batch query
% \newcommandx{\vI}[2]{\v{1}} %Indicator vector
% \newcommandx{\Sim}[2]{\A{S}} %Simulator
% \newcommandx{\Ser}[2]{\A{S}} %Server
% \newcommandx{\SimC}{\A{C}\ast} %The adversarial client
% \newcommandx{\SimS}{\A{S}\ast} %The adversarial server
\newcommandx{\sCs}{\set{C}\ast} %The challenge client set
\newcommandx{\sSs}{\set{S}\ast} %The challenge client set
% \newcommandx{\Ch}[1][1=]{\A[h][#1]{C}} %The challenger
\newcommandx{\abort}{\TK{abort}} %Abort event
% \newcommandx{\cabort}[2]{\cEv{abort}} %Complement of abort
\newcommandx{\fail}[2]{\TK{fail}\xspace}
% \newcommandx{\T}[1][1=]{\Table[#1]{L}} %Default table
% \renewcommandx{\r}{r} %randomiser
\newcommandx{\erf}{\f{erf}} %Error function for Gaussians
%END

%%%%%%%%%%%%%%%%%%%%%%%%%%%%%%%%%%%%%%%%%%%%%%%%%%%%%%%%%%%%%%%%%%%%%%%%%%%%%%%%%%%%%%%%%%%%%%%%%%%%%%%%%%
%PROVABLE SECURITY
%BEGIN
%Basic
\newcommandx{\Adv}{\tB{Adv}} %Adversarial advantage
\newcommandx{\view}{\tB{view}} %view of a party in MPC, simulation
\newcommandx{\iView}[4][1=,2=]{\set{\view\script[#1][#2]\left(#3\right)}_{#4}} %View in MPC, simulation
\renewcommandx{\output}{\tB{output}} %output of a party in MPC
\newcommandx{\real}{\tB{real}} %real view/output distribution of a party in MPC
\newcommandx{\ideal}{\tB{ideal}} %ideal view/output distribution of a party in MPC
\newcommandx{\hybrid}{\tB{hybrid}} %hybrid view/output distribution of a party in MPC
\renewcommandx{\Game}[1]{\mathcal{#1}} %Game


%Security Assumptions
\newcommandx{\Assumption}[1]{\TC{#1}\xspace} %Security assumption
\newcommandx{\RSA}{\Assumption{RSA}}
\newcommandx{\ADH}{\Assumption{DH}}
\newcommandx{\ADL}{\Assumption{DL}}
\newcommandx{\DLP}{\Assumption{DLP}}
\newcommandx{\OMODL}{\Assumption{OM-DL}}
\newcommandx{\DDH}{\Assumption{DDH}}
\newcommandx{\GDH}{\Assumption{GDH}}
\newcommandx{\OMRSA}{\Assumption{OM-RSA}}
\newcommandx{\pOMRSA}[1]{(#1)\Assumption{-OM-RSA}}  
\newcommandx{\OMGDH}{\Assumption{OM-GDH}}
\newcommandx{\pOMGDH}[1]{(#1)-\Assumption{OM-GDH}}  
\newcommandx{\SVP}{\Assumption{SVP}}
\newcommandx{\SIVP}{\Assumption{SIVP}} 
\newcommandx{\LWE}{\Assumption{LWE}}

%Security Models
\newcommandx{\Model}[1]{\TC{#1}\xspace} %Security model
\newcommandx{\sID}{\Model{sID}}
\newcommandx{\wID}{\Model{wID}}
\newcommandx{\CCA}{\Model{CCA}}
\newcommandx{\CPA}{\Model{CPA}}
\newcommandx{\ROM}{\Model{ROM}} %ROM model
\newcommandx{\pROM}{\Model{pROM}} %Parallel ROM model
\newcommandx{\EUCMA}{\Model{EU-CMA}}
\newcommandx{\EUNMA}{\Model{EU-NMA}}
\newcommandx{\EUGCMA}{\Model{EU-GCMA}}
\newcommandx{\EUIDCMA}{\Model{EU-ID-CMA}}
\newcommandx{\EUsIDCMA}{\Model{EU-sID-CMA}}
\newcommandx{\EUwIDCMA}{\Model{EU-wID-CMA}}
\newcommandx{\IND}{\Model{IND}}
\newcommandx{\SIM}{\Model{SIM}}
\newcommandx{\INDCU}{\Model{IND-CU}}
\newcommandx{\INDSU}{\Model{IND-SU}}
\newcommandx{\INDCP}{\Model{IND-CP}}
\newcommandx{\INDSP}{\Model{IND-SP}}
\newcommandx{\SIMSP}{\Model{SIM-SP}}
\newcommandx{\pINDSP}[1]{\Model{#1-IND-SP}}
\newcommandx{\nINDSP}{\pINDSP{n}}
\newcommandx{\pINDCP}[1]{\Model{#1-IND-CP}}
\newcommandx{\nINDCP}{\pINDCP{n}}
\newcommandx{\pSIMSP}[1]{\Model{#1-SIM-SP}}
\newcommandx{\pSIMCP}[1]{\Model{#1-SIM-CP}}
\newcommandx{\SIMCU}{\Model{SIM-CU}}
\newcommandx{\SIMSU}{\Model{SIM-SU}}
\newcommandx{\nSIMSP}{\Model{n-SIM-SP}}
\newcommandx{\nSIMCP}{\Model{n-SIM-CP}}
% \newcommandx{\PRECPA}{\Model{PRE-CPA}}
% \newcommandx{\PRECCA}{\Model{PRE-CCA}}
% \newcommandx{\PRKCPA}{\Model{PRK-CPA}}
% \newcommandx{\PRKCCA}{\Model{PRK-CCA}}
% \newcommandx{\sPRECPA}{\Model{sPRE-CPA}}
% \newcommandx{\sPRECCA}{\Model{sPRE-CCA}}
% \newcommandx{\sPRKCPA}{\Model{sPRK-CPA}}
% \newcommandx{\sPRKCCA}{\Model{sPRK-CCA}}

%Oracles
% Oracles
\newcommandx{\oracle}{\Algorithm{O}}
\newcommandx{\extract}{\varepsilon} %extract oracle
\newcommandx{\sign}{s} %Sign oracle
% \newcommandx{\view}{v} %View Oracle
\newcommandx{\OS}{\oracle @[\sign]} %Signing-oracle
\newcommandx{\OK}{\oracle @[\extract]} %Key-extract oracle
\newcommandx{\OV}{\oracle @[v]} %View-oracle
\newcommandx{\OSE}{\oracle @[\{\sign,\extract\}]} %Both extract and signing oracle
\newcommandx{\OSEH}{\oracle @[\{\sign,\HB{\extract}\}]} 
\newcommandx{\ORSA}[1][1=]{\oracle @[\RSA,#1]}
\newcommandx{\OC}{\oracle @[c]} %Challenge oracle
\newcommandx{\OaDH}{\oracle @[\DH,\alpha]}
\newcommandx{\ObDH}{\oracle @[\DH,\beta]}
\newcommandx{\ODDH}{\oracle @[\DDH]}
\newcommandx{\ODH}{\oracle @[\DH]}


%MPC-related Commands
\newcommandx{\TTP}{\TP{TTP}\xspace} %Trusted third party
\newcommandx{\HbC}{\Model{HbC}} %Honest but curious

% Bound on queries
\newcommandx{\qH}{q_\O{H}}
\newcommandx{\qG}{q_\O{G}}
\newcommandx{\qR}{q_\O{R}}
\newcommandx{\qC}{q_\O{c}}
\newcommandx{\qE}{q_\extract} %Bound on extract query
\newcommandx{\qS}{q_\sign} %Bound on sign query
\newcommandx{\QH}[1]{q_{\O{H}_{#1}}}
\newcommandx{\qCh}{q_\O{Ch}}

%Key-type fonts
\newcommandx{\id}{\TK{id}\xspace}
\newcommandx{\kk}{\TK{k}\xspace}  %Normal key
\newcommandx{\usk}{\TK{usk}\xspace} % User secret key
\newcommandx{\PP}{\TK{pp}\xspace} %Public parameters
\newcommandx{\param}{\TK{param}\xspace} %Public parameters, alternative
\newcommandx{\mpk}{\TK{mpk}\xspace} %Master public key
\newcommandx{\msk}{\TK{msk}\xspace} %Master secret key
\newcommandx{\kp}{\TK{pk}\xspace} %Public key
\newcommandx{\ks}{\TK{sk}\xspace} %Secret key
\newcommandx{\kr}{\TK{rk}\xspace} %Re-encryption key
\newcommandx{\ke}{\TK{ek}\xspace} %Evaluation key
\newcommandx{\kt}{\TK{td}\xspace} %Trapdoor
%END

\newcommand{\hati}{\hat{\hspace{-.3ex}\textit{\i}}\xspace} %i hat


% \newcommandx{\hati}{\hat\textit{\i}\xspace}
%%%%%%%%%%%%%%%%%%%%%%%%%%%%%%%%%%%%%%%%%%%%%%%%%%%%%%%%%%%%%%%%%%%%%%%%%%%%%%%%%%%%%%%%%%%%%%%%%%%%%%%%%%
%COMPLEXITY THEORY
%BEGIN
%Complexity functions

\NewDocumentCommand\BO{sm}{%Big-Oh
  \IfBooleanTF#1%
    {\tF{O}\left(#2\right)}%with braces
    {\tF{O}(#2)}
}
\NewDocumentCommand\BOL{sm}{%Big-Oh ignoring Logarithmic factors
  \IfBooleanTF#1%
    {\tilde{\tF{O}}\left(#2\right)}%with braces
    {\tilde{\tF{O}}(#2)}
}
\NewDocumentCommand\sO{sm}{%Small-Oh
  \IfBooleanTF#1%
    {\tF{o}\left(#2\right)}%with braces
    {\tF{o}(#2)}
}
\NewDocumentCommand\BW{sm}{%Big-Omega
  \IfBooleanTF#1%
    {\Omega\left(#2\right)}%with braces
    {\Omega(#2)}
}
\NewDocumentCommand\BWL{sm}{%Big-Omega ignoring Logarithmic factors
  \IfBooleanTF#1%
    {\tilde{\Omega}\left(#2\right)}%with braces
    {\tilde{\Omega}(#2)}
}
\NewDocumentCommand\sW{sm}{%Small-omega
  \IfBooleanTF#1%
    {\omega\left(#2\right)}%with braces
    {\omega(#2)}
}
\NewDocumentCommand\BQ{sm}{%Big-theta
  \IfBooleanTF#1%
    {\Theta\left(#2\right)}%with braces
    {\Theta(#2)}
}

%Classes
\newcommandx{\Hard}{\TP{-}\Class{H}} %-hard
\newcommandx{\co}{\Class{co}\text{-}} %co- 
\newcommandx{\Co}{\text{-}\Class{C}} %-complete
\newcommandx{\poly}{\Class{poly}\xspace} %Polynomial functions
\newcommandx{\polylog}{\Class{polylog}\xspace} %Poly-log functions
\newcommandx{\negl}{\Class{negl}\xspace} %Negligible function
\newcommandx{\expn}{\Class{exp}\xspace} %Exponential function

%END

%%%%%%%%%%%%%%%%%%%%%%%%%%%%%%%%%%%%%%%%%%%%%%%%%%%%%%%%%%%%%%%%%%%%%%%%%%%%%%%%%%%%%%%%%%%%%%%%%%%%%%%%%%
%TEXT
%BEGIN
%Superscripts
\newcommandx{\sth}{\script[][\tF{th}]} %nth
\newcommandx{\sst}{\script[][\tF{st}]} %1st
\newcommandx{\snd}{\script[][\tF{nd}]} %2nd
\newcommandx{\srd}{\script[][\tF{rd}]} %3rd

% Texts in equations
\newcommandx{\tM}[1]{~\textit{#1}~}
\newcommandx{\MST}{\tM{\ST}}
\newcommandx{\ecoma}{,~}
\newcommandx{\eand}{\textrm{~~and~~}}   %GF: I don't know what \tM did, but it made the "and" italic even in non-italic text
\newcommandx{\eor}{\tM{or}}
\newcommandx{\ewith}{\tM{with}}
%\newcommandx{\espace}[1][4]{\foreach \n in {0,...,#1}{~}}
\newcommandx{\comment}[1]{\TP{(#1)}} %comment in algorithm
\newcommandx{\sComment}[1]{~~\TP{(#1)}} %comment with spaces -- for equations
\newcommandx{\mComment}[2]{~~\TP{(#1 }#2\TP{)}} %comment in math mode

% Etc
\makeatletter 
\newcommandx{\eg}{\TP{e.g}\@ifnextchar.{}{.\@}\xspace}
\newcommandx{\Eg}{\TP{E.g}\@ifnextchar.{}{.\@}\xspace}
\newcommandx{\wrt}{\TP{w.r.t}\@ifnextchar.{}{.\@}\xspace}
\newcommandx{\af}{\TP{au fond}\xspace}
\newcommandx{\etal}{\TP{et al}\@ifnextchar.{}{.\@}\xspace}
\newcommandx{\vs}{\TP{vs}\@ifnextchar.{}{.\@}\xspace}
\newcommandx{\etc}{\TP{etc}\@ifnextchar.{}{.\@}\xspace}
\newcommandx{\viz}{\TP{viz}\@ifnextchar.{}{.\@}\xspace}
\newcommandx{\ala}{\TP{\`{a} la}\xspace}
\newcommandx{\iaof}{\TP{iff}\xspace}
\newcommandx{\resp}{resp\@ifnextchar.{}{.\@}\xspace}
\newcommandx{\perse}{\TP{per se}\xspace}
\newcommandx{\df}{\TP{de facto}\xspace}
\newcommandx{\Dr}{Dr\@ifnextchar.{}{.\@}\xspace}
\newcommandx{\Ms}{Ms\@ifnextchar.{}{.\@}\xspace}
\newcommandx{\Mr}{Mr\@ifnextchar.{}{.\@}\xspace}
\newcommandx{\Mrs}{Mrs\@ifnextchar.{}{.\@}\xspace}
\newcommandx{\Prof}{Prof\@ifnextchar.{}{.\@}\xspace}
\newcommandx{\QED}{\TP{Q.E.D\@ifnextchar.{}{.\@}\xspace}}
\newcommandx{\QEF}{\TP{Q.E.F\@ifnextchar.{}{.\@}\xspace}}
\newcommandx{\st}{\TP{s.t\@ifnextchar.{}{.\@}\xspace}} %such that
\newcommandx{\ie}{\TP{i.e}\@ifnextchar.{}{.\@}\xspace}
\newcommandx{\cf}{\TP{cf}\@ifnextchar.{}{.\@}\xspace}
\newcommandx{\Ie}{\TP{I.e}\@ifnextchar.{}{.\@}\xspace}
\newcommandx{\iid}{\TP{i.i.d}\@ifnextchar.{}{.\@}\xspace}
\makeatother

%Case, type, round, flag...
\ifx \slide \undefined
  \newcommandx{\type}[1]{\TC{type~#1}\xspace}
  \newcommandx{\round}[1]{\TC{round~#1}\xspace}
  \newcommandx{\flag}{\TC{flag}\xspace}
  \newcommandx{\Case}[1]{\TC{case~#1}\xspace}  
\else
  \newcommandx{\type}[1]{\TK{type~#1}\xspace}
  \newcommandx{\round}[1]{\TK{round~#1}\xspace}
  \newcommandx{\flag}{\TK{flag}\xspace}
  \newcommandx{\Case}[1]{\TK{case~#1}\xspace}
\fi
%END

%%%%%%%%%%%%%%%%%%%%%%%%%%%%%%%%%%%%%%%%%%%%%%%%%%%%%%%%%%%%%%%%%%%%%%%%%%%%%%%%%%%%%%%%%%%%%%%%%%%%%%%%%%%
%REFERENCES AND HYPERREFS
%BEGIN
%Defining backref
\ifx \lncs \undefined
  \usepackage[pagebackref=true,hidelinks]{hyperref}
  \hypersetup{
    pdfborder=0 0 0,
    colorlinks   = true,
    citecolor    = violet,
    linkcolor	=  BrickRed
%     anchorcolor [black]
%     citecolor [green]
%     filecolor [cyan]
%     menucolor [red]
%     runcolor [cyan - same as file color]
%     urlcolor [magenta]
%     allcolors -- use this if you want to set all links to the same color
%     citebordercolor [rgb 0 1 0]
%     filebordercolor [rgb 0 .5 .5]
%     linkbordercolor [rgb 1 0 0]
%     menubordercolor [rgb 1 0 0]
%     urlbordercolor [rgb 0 1 1]
%     runbordercolor [rgb 0 .7 .7]
%     allbordercolors
  }
  \renewcommand*{\backref}[1]{}
  \renewcommand*{\backrefalt}[4]{
  \ifcase #1
    (Not cited.)
  \or
    (Cited on page~#2.)
  \else
    (Cited on pages~#2.)
  \fi}
  \renewcommand*{\backrefsep}{, }
  \renewcommand*{\backreftwosep}{ and~}
  \renewcommand*{\backreflastsep}{ and~}
% \else
%   \usepackage[hidelinks]{hyperref}
\else
  \usepackage[breaklinks,colorlinks,citecolor=gray,linkcolor=blue]{hyperref}
  \usepackage{breakcites} 
\fi

\newcommandx{\footnoteref}[1]{\textsuperscript{\ref{#1}}}

%Blank Footnote
\newcommandx\blfootnote[1]{
  \begingroup
  \renewcommand\thefootnote{}\footnote{#1}
  \addtocounter{footnote}{-1}
  \endgroup
}
 
%Bibliography style
% \bibliographystyle{alpha}

\newcommandx{\Ref}[1]{\TR{\ref{#1}}} %General refernce
\newcommandx{\RP}[1]{Page~\pageref{#1}} %Page
\newcommandx{\RE}[1]{{\text(\ref{#1})}} %Equation
% \ifx \lncs \undefined
   \newcommandx{\RF}[1]{\TR{Figure~\ref{#1}}} %Figure
   \newcommandx{\RT}[1]{\TR{Theorem~\ref{#1}}} %Theorem
%    \else\newcommandx{\RF}[1]{\TR{Figure.~\ref{#1}}} %Figure
% 	 \newcommandx{\RT}[1]{\TR{Thm.~\ref{#1}}} %Theorem
% \fi
\newcommandx{\RL}[1]{\TR{Lemma~\ref{#1}}} %Lemma
\newcommandx{\RC}[1]{\TR{Claim~\ref{#1}}} %Claim
% \ifx \lncs \undefined
   \newcommandx{\RCn}[1]{\TR{Construction~\ref{#1}}} %Construction
%    \else\newcommandx{\RCn}[1]{\TR{Constr.~\ref{#1}}} %Construction
% \fi
\newcommandx{\Rp}[1]{\TR{Proposition~\ref{#1}}} %Proposition
\newcommandx{\Rc}[1]{\TR{Corollary~\ref{#1}}} %Corollary
% \ifx \lncs \undefined
   \newcommandx{\RD}[1]{\TR{Definition~\ref{#1}}} %Definition
%    \else\newcommandx{\RD}[1]{\TR{Def.~\ref{#1}}} %Definition
% \fi
\newcommandx{\RS}[1]{\TR{§\ref{#1}}} %Section
\newcommandx{\RI}[1]{\TR{Item~\ref{#1}}} %Item
\newcommandx{\RCh}[1]{\TR{Chapter~\ref{#1}}} %Chapter
\newcommandx{\RR}[1]{\TR{Remark~\ref{#1}}} %Remark
\newcommandx{\RO}[1]{\TR{Observation~\ref{#1}}} %Observation
\newcommandx{\Rt}[1]{\TR{Table~\ref{#1}}} %Table
\newcommandx{\RFo}[1]{\TR{Footnote~\ref{#1}}} %Footnote
\newcommandx{\Ra}[1]{\TR{Appendix~\ref{#1}}} %Appendix
\newcommandx{\RA}[1]{\TR{Algorithm~\ref{#1}}} %Algorithm
\newcommandx{\RAs}[1]{\TR{Assumption~\ref{#1}}} %Assumption
\newcommandx{\RG}[1]{\TR{Game~\ref{#1}}} %Game
\newcommandx{\RPr}[1]{\TR{Procedure~\ref{#1}}} %Procedure

%END

%%%%%%%%%%%%%%%%%%%%%%%%%%%%%%%%%%%%%%%%%%%%%%%%%%%%%%%%%%%%%%%%%%%%%%%%%%%%%%%%%%%%%%%%%%%%%%%%%%%%%%%%%%%
%ALGORITHM
%BEGIN
\newcommand{\subpara}[1]{\noindent\emph{#1}}
% \newcommand{\paragraph}[1]{\paragraph\textit{#1}}
% \newcommand{\paragraph}[1]{\vspace{-4mm}\paragraph\textit{#1}}
%TO-do: change the inline If
% \algnewcommand{\IIf}[1]{\State\algorithmicif\ #1\ \algorithmicthen}
% \algnewcommand{\EndIIf}{\unskip\ \algorithmicend\ \algorithmicif}
% \renewcommand{\Comment}[1]{~~/\hspace{-1mm}/\TS{#1}}
\algnewcommand{\IFor}[1]{\State\algorithmicfor\ #1\ \algorithmicdo}
\algnewcommand{\EndIFor}{\unskip\ \algorithmicend\ \algorithmicfor}
\algnewcommand{\IIf}[1]{\State\algorithmicif\ #1\ \algorithmicthen\ }
\algnewcommand{\EndIIf}{\unskip\ \algorithmicend\ \algorithmicif}
% \newcommand{\LineIf}[2]{%
%   \State \algorithmicif\ {#1} \algorithmicthen\ {#2}
% }
\newcommand{\LineIfElse}[3]{%
  \State \algorithmicif\ {#1} \algorithmicthen\ {#2}
  \State \algorithmicelse\ {#3} \algorithmicend\ \algorithmicif
}
\newcommand{\CompactIfElse}[3]{%
   \State \algorithmicif\ {#1} \algorithmicthen\ {#2} \algorithmicelse\ {#3} \algorithmicend\ \algorithmicif
}
\newcommandx{\algorithmicow}{\TB{otherwise}}
% \newenvironment{experiment}[1][htb]
%   {\renewcommand{\algorithmcfname}{Experiment}% Update algorithm name
%    \begin{algorithm}[#1]%
%   }{\end{algorithm}}

% Swith-case
\algnewcommand\algorithmicswitch{\TB{switch}}
\algnewcommand\algorithmiccase{\TB{case}}
\algnewcommand\algorithmicassert{\TB{assert}}
\algnewcommand\Assert[1]{\State \algorithmicassert(#1)}
\algdef{SE}[SWITCH]{Switch}{EndSwitch}[1]
    {\algorithmicswitch\ #1\ \algorithmicdo}
    {\algorithmicend\ \algorithmicswitch}
\algdef{SE}[CASE]{Cases}{EndCases}[1]
    {\algorithmiccase\ #1}
    {\algorithmicend\ \algorithmiccase}
%\algtext*{EndSwitch}
%\algtext*{EndCases}
\newcommand{\LineCase}[2]{%
  \State \algorithmiccase\ {#1}: {#2}
}
\newcommandx{\continue}{\TB{continue}}
% \renewcommandx{\break}{\TB{break}} %Caused the ``break'' probelems!
\newcommandx{\return}{\algorithmicreturn}

\makeatletter
\def\ALG@special@indent{%
    \ifdim\ALG@thistlm=0pt\relax
        \hskip-\leftmargin
    \else
        \hskip\ALG@thistlm
    \fi
}
\newcommand{\IState}[1]{\item[]\noindent\ALG@special@indent #1}
 
\makeatother
%END

%%%%%%%%%%%%%%%%%%%%%%%%%%%%%%%%%%%%%%%%%%%%%%%%%%%%%%%%%%%%%%%%%%%%%%%%%%%%%%%%%%%%%%%%%%%%%%%%%%%%%%%%%%%%%%%%%%%%%%%%%%%%
%TIKZ
%BEGIN
  \makeatletter
  \newcommand{\thickhline}{%
    \noalign {\ifnum 0=`}\fi \hrule height 1.25pt
    \futurelet \reserved@a \@xhline
  }
  \newcolumntype{"}{@{\hskip\tabcolsep\vrule width 1.25pt\hskip\tabcolsep}}
  \makeatother
%   \newcolumntype{"}{@{\vrule width 1pt}}
  
  \definecolor{light-gray}{gray}{0.75}
  \definecolor{lighter-gray}{gray}{0.85}
  \definecolor{lightest-gray}{gray}{0.99}

  \tikzstyle{rbox}=[draw, minimum size=2em, thick]
  \tikzstyle{rbbox}=[draw, minimum size=2em,white]
  \tikzstyle{box}=[draw, minimum size=3em, thick]
  \tikzstyle{ibox}=[draw, minimum size=3.75em, thick]
  \tikzstyle{mbox}=[draw, minimum size=4.5em, thick]
  \tikzstyle{bbox}=[draw, minimum width=4.5em,minimum height=3em, thick]
  \tikzstyle{lbox}=[draw, minimum size=6em, thick]
  \tikzstyle{llbox}=[draw, minimum size=9em, thick]
  \tikzstyle{blbox}=[draw, minimum width=8em, minimum height=6em, thick]
  \tikzstyle{sbox}=[draw, minimum width=4.5em, minimum height=6em, thick]
  \tikzstyle{buffer}=[draw,white]
  \tikzstyle{secure} = [pin edge={to-,thick,black}]
  \tikzstyle{public} = [pin edge={to-,thick,black,dashed}]
  \tikzstyle{highlight} = [preaction={draw,light-gray,-,double=light-gray,double distance=4}]
  \tikzstyle{superimpose} = [fill=white]
  \tikzstyle{ROC} = [draw, circle, minimum size=.5cm, node distance=1.75cm]
  
  \tikzset{>=latex}   

  \newcommand{\HCR}[1]{{\color{red!66} #1}}
  \newcommand{\HCLR}[1]{{\color{red!66} #1}}
  \newcommand{\HCB}[1]{{\color{blue!66} #1}}
  \newcommand{\HCLB}[1]{{\color{blue!66} #1}}
  \newcommand{\HCG}[1]{{\color{green!66} #1}}
  \newcommand{\HCLG}[1]{{\color{green!50!black!66} #1}}
  \newcommand{\DHDG}[1]{{\color{gray} #1}}
  \newcommand{\DHB}[1]{{\color{black} #1}}
  \newcommand{\DHG}[1]{{\color{light-gray} #1}}
  \newcommand{\HCV}[1]{{\color{violet} #1}}
  \newcommand{\DHLG}[1]{{\color{lightest-gray} #1}}
  \newcommand{\hl}[1]{\colorbox{yellow}{#1}}
  
  \newcommand{\bdot}{\bullet}
  \newcommand{\hdot}{\circ}
  \newcommand{\rdot}{\HCLR{\bullet}}
  \newcommand{\bldot}{\HCLB{\bullet}}
  \newcommand{\gdot}{\HCLG{\bullet}}
  \newcommand{\ldot}{\DHLG{\bullet}}
%   \newcommand{\ndot}{\ldot\ldot\ldot\ldot}

\newcommand*\circled[1]{\tikz[baseline=(char.base)]{
            \node[shape=circle,draw,inner sep=0.1em,thick] (char) {#1};}}
\newcommand*\pebble{\tikz[baseline=(char.base)]{
            \node[shape=circle,draw,inner sep=0.1em,fill] (char) {1};}}

\newcommand{\ntikzmark}[2]{#2\thinspace\tikz[overlay,remember picture,baseline=(#1.base)]{\node[inner sep=0pt] (#1) {};}}

\newcommand{\makebrace}[3]{%
    \begin{tikzpicture}[overlay, remember picture]
        \draw [decoration={brace,amplitude=0.5em},decorate,thick] let \p1=(#1), \p2=(#2) in ({max(\x1,\x2)}, {\y1+0.8em}) -- node[right=0.6em] {#3} ({max(\x1,\x2)}, {\y2});
    \end{tikzpicture}
}

\newcommand{\callout}[3]{
  \begin{tikzpicture}[overlay, remember picture]
    \node[overlay,rectangle callout,callout relative pointer={(-1em,-1em)},fill=blue!33, above of=#2, node distance=#3] {#1};
  \end{tikzpicture}
}
\newcommand{\point}[3]{
  \begin{tikzpicture}[overlay, remember picture]
    \node[overlay, above of=#2, node distance=#3] (X) {#1};
    \draw[->,blue!66] (X.south) -- ([xshift=-0.5em]#2.north);
  \end{tikzpicture}
}

\tikzset{markplace/.style=
 {rectangle callout,
  fill=orange!66,
  callout absolute pointer={#1},
  at={#1},
  above=1em
  }}


\newcommand\redout{\bgroup\markoverwith
{\textcolor{red}{\rule[0.5ex]{2pt}{0.8pt}}}\ULon}

% \tikzset{
%   invisible/.style={opacity=0},
%   visible on/.style={alt=#1{}{invisible}},
%   alt/.code args={<#1>#2#3}{%
%     \alt<#1>{\pgfkeysalso{#2}}{\pgfkeysalso{#3}} % \pgfkeysalso doesn't change the path
%   },
% }    

%END

%%%%%%%%%%%%%%%%%%%%%%%%%%%%%%%%%%%%%%%%%%%%%%%%%%%%%%%%%%%%%%%%%%%%%%%%%%%%%%%%%%%%%%%%%%%%%%%%%%%%%%%%%%%
%MISCELLANEOUS
%BEGIN
\renewcommand{\arraystretch}{1.75}
\setcounter{secnumdepth}{3}
% \makenomenclature
% \renewcommand{\nomname}{List of Abbreviations}
% \nomitemsep{2em}
%END

% \begin{inparaenum}[\itshape i\upshape)]
%   \item 
% \end{inparaenum}

%Define '@' as subscript/superscript operator
%IMPORTANT that this command is at the bottom.
\begingroup
\catcode`@=\active
\global\let @=\script
\endgroup
\AtBeginDocument{\mathcode`@=\string"8000} % \string is to avoid problems with babel


\let\subsubsectionTEMP\subsubsection
\renewcommand{\subsubsection}[1]{\vspace{-1\smallskipamount}\subsubsectionTEMP{#1}}
\newcommand{\spc}{\vspace{-\smallskipamount}}

\usepackage{rotating}


% Controlling the margin
% \usepackage[margin=1.25in]{geometry}
% \linespread{0.9}
\bibliographystyle{alpha}
\pagestyle{plain}

%Local macros
\newcommandx{\NN}{\mathbb{N}}%Natural numbers

\newcommand{\rbullet}{\color{red} \bullet}
\newcommand{\bbullet}{\color{blue} \bullet}

\newcommand{\kpnote}[1]{{\textcolor{green}{KP: #1}}}
\newcommand{\cknote}[1]{{\textcolor{blue}{CK: #1}}}
\newcommand{\kknote}[1]{{\textcolor{red}{KK: #1}}}
\newcommand{\hanote}[1]{{\textcolor{orange}{HA: #1}}}

%opening
\title{PoSW from Skip List}
\author{}
% \institute{}

\begin{document}
\maketitle
\begin{abstract}
\end{abstract}

\section{Introduction}
  \subsection{PoSW}
    \begin{enumerate}
      \item PoSW vs PoW vs Time-lock puzzles
      \item Original construction from depth-robust graphs
    \end{enumerate}

  \subsection{Related Work}
    \begin{enumerate}
      \item Time-release crypto \cite{RSW00,BGJ+15,May93,MMV11}
      \item Original construction \cite{MMV13} using depth-robust graphs  \cite{MMV13,EGS75}
      \item \cite{CP18} construction doesn't require depth-robustness
    \end{enumerate}

  \subsection{Our Contribution}
    \begin{enumerate}
      \item Intuitive construction of POSW based on skip lists \cite{Pug90}
      \item Larger gap in proof generation and verification using PRPs and sloth hash function \cite{LW17}
      \item PRP used instead of a hash function and the input to it is squared before
    \end{enumerate}

  
\section{Preliminaries}
  \subsection{Formal Definitions}
    \begin{enumerate}
      \item PoSW 
      \item 
    \end{enumerate}

  \subsection{Skip list}
    \begin{enumerate}
      \item \cite{Pug90}
      \item randomised \vs deterministic (binary)
      \item Figure
    \end{enumerate}
    
    \begin{figure}
      \centering
      \begin{tikzpicture}[
	node distance=1.5em,
	pointer/.style={draw,thick,minimum width=1.5em,minimum height=1.5em},
	data/.style={draw,thick,minimum width=1.5em,minimum height=1.5em},
	p/.style={->,thick}]
	
	\node[pointer] (01) {$\bullet$};
	\node[data,right of=01,xshift=1em] (10) {$3$};
	\node[pointer,right of=10] (11) {$\bullet$};
	\node[data,right of=11,xshift=1em] (20) {$6$};
	\node[pointer,right of=20] (21) {$\bullet$};
	\node[data,right of=21,xshift=1em] (30) {$7$};
	\node[pointer,right of=30] (31) {$\bullet$};
	\node[data,right of=31,xshift=1em] (40) {$9$};
	\node[pointer,right of=40] (41) {$\bullet$};
	\node[data,right of=41,xshift=1em] (50) {\small $12$};
	\node[pointer,right of=50] (51) {$\bullet$};
	\node[data,right of=51,xshift=1em] (60) {\small $17$};
	\node[pointer,right of=60] (61) {$\bullet$};
	\node[data,right of=61,xshift=1em] (70) {\small $19$};
	\node[pointer,right of=70] (71) {$\bullet$};
	\node[data,right of=71,xshift=1em] (80) {\small $21$};
	\node[pointer,right of=80] (81) {$\bullet$};
	\node[data,right of=81,xshift=1em] (90) {\small $25$};
	\node[pointer,right of=90] (91) {$\bullet$};
	\node[data,right of=91,xshift=1em] (100) {$\bot$};
	
	\foreach \i/\j in {0/1,1/2,2/3,3/4,4/5,5/6,6/7,7/8,8/9,9/10}{
	  \draw[p] (\i1.center) -> (\j0);
	}
      \end{tikzpicture}

      \bigskip
      
      \begin{tikzpicture}[
	pointer/.style={draw,thick,minimum width=1.5em,minimum height=1.5em},
	data/.style={draw,thick,minimum width=1.5em,minimum height=1.5em},
	holder/.style={minimum width=1.5em,minimum height=1.5em},
	p/.style={->,thick},
	node distance=1.5em]
	%State 0
	\node[pointer] (001) {$\bullet$};
	\node[pointer,above of=001] (011) {$\bullet$};
	
	%State 1
	\node[data,right of=001,xshift=1em] (100) {$3$};
	\node[pointer,right of=100] (101) {$\bullet$};
	
	%State 000010
	\node[holder,right of=101,xshift=1em] (200) {};
	\node[holder,above of=200] (210) {};
	\node[pointer,right of=200] (201) {$\bullet$};
	\node[pointer,above of=201] (211) {$\bullet$};
	\node[data,above of=200,minimum height=3em,yshift=-0.75em] (200d) {$6$};
	
	
	%State 3
	\node[data,right of=201,xshift=1em] (300) {$7$};
	\node[pointer,right of=300] (301) {$\bullet$};
	
	%State 4
	\node[holder,right of=301,xshift=1em] (400) {};
	\node[holder,above of=400] (410) {};
	\node[pointer,right of=400] (401) {$\bullet$};
	\node[pointer,above of=401] (411) {$\bullet$};
	\node[data,above of=400,minimum height=3em,yshift=-0.75em] (400d) {$9$};

	%State 5
	\node[data,right of=401,xshift=1em] (500) {\small $12$};
	\node[pointer,right of=500] (501) {$\bullet$};
	
	%State 6
	\node[holder,right of=501,xshift=1em] (600) {};
	\node[holder,above of=600] (610) {};
	\node[pointer,right of=600] (601) {$\bullet$};
	\node[pointer,above of=601] (611) {$\bullet$};
	\node[data,above of=600,minimum height=3em,yshift=-0.75em] (600d) {\small $17$};
	
	%State 7
	\node[data,right of=601,xshift=1em] (700) {\small $19$};
	\node[pointer,right of=700] (701) {$\bullet$};
	
	%State 8
	\node[holder,right of=701,xshift=1em] (800) {};
	\node[holder,above of=800] (810) {};
	\node[pointer,right of=800] (801) {$\bullet$};
	\node[pointer,above of=801] (811) {$\bullet$};
	\node[data,above of=800,minimum height=3em,yshift=-0.75em] (800d) {\small $21$};
	
	%State 9
	\node[data,right of=801,xshift=1em] (900) {\small $25$};
	\node[pointer,right of=900] (901) {$\bullet$};
	
	%State 10
	\node[holder,right of=901,xshift=1em] (1000) {};
	\node[holder,above of=1000] (1010) {};
	\node[data,above of=1000,minimum height=3em,yshift=-0.75em] (1000d) {$\bot$};

	%Level 0 edges
	\foreach \i/\j in {0/1,1/2,2/3,3/4,4/5,5/6,6/7,7/8,8/9,9/10}{
	  \draw[p] (\i01.center) -> (\j00);
	}
	
	%Level 1 edges
	\foreach \i/\j in {0/2,2/4,4/6,6/8,8/10}{
	  \draw[p] (\i11.center)-> (\j10);
	}
      \end{tikzpicture}
      
      \bigskip
      
      \begin{tikzpicture}[
	pointer/.style={draw,thick,minimum width=1.5em,minimum height=1.5em},
	data/.style={draw,thick,minimum width=1.5em,minimum height=1.5em},
	holder/.style={minimum width=1.5em,minimum height=1.5em},
	p/.style={->,thick},
	node distance=1.5em]
	%State 0
	\node[pointer] (001) {$\bullet$};
	\node[pointer,above of=001] (011) {$\bullet$};
	\node[pointer,above of=011] (021) {$\bullet$};
	
	%State 1
	\node[data,right of=001,xshift=1em] (100) {$3$};
	\node[pointer,right of=100] (101) {$\bullet$};
	
	%State 000010
	\node[holder,right of=101,xshift=1em] (200) {};
	\node[holder,above of=200] (210) {};
	\node[pointer,right of=200] (201) {$\bullet$};
	\node[pointer,above of=201] (211) {$\bullet$};
	\node[data,above of=200,minimum height=3em,yshift=-0.75em] (200d) {$6$};
	
	
	%State 3
	\node[data,right of=201,xshift=1em] (300) {$7$};
	\node[pointer,right of=300] (301) {$\bullet$};
	
	%State 4
	\node[holder,right of=301,xshift=1em] (400) {};
	\node[holder,above of=400] (410) {};
	\node[holder,above of=410] (420) {};
	\node[pointer,right of=400] (401) {$\bullet$};
	\node[pointer,above of=401] (411) {$\bullet$};
	\node[pointer,above of=411] (421) {$\bullet$};
	\node[data,above of=400,minimum height=4.5em] (400d) {$9$};

	%State 5
	\node[data,right of=401,xshift=1em] (500) {\small $12$};
	\node[pointer,right of=500] (501) {$\bullet$};
	
	%State 6
	\node[holder,right of=501,xshift=1em] (600) {};
	\node[holder,above of=600] (610) {};
	\node[pointer,right of=600] (601) {$\bullet$};
	\node[pointer,above of=601] (611) {$\bullet$};
	\node[data,above of=600,minimum height=3em,yshift=-0.75em] (600d) {\small $17$};
	
	%State 7
	\node[data,right of=601,xshift=1em] (700) {\small $19$};
	\node[pointer,right of=700] (701) {$\bullet$};
	
	%State 8
	\node[holder,right of=701,xshift=1em] (800) {};
	\node[holder,above of=800] (810) {};
	\node[holder,above of=810] (820) {};
	\node[pointer,right of=800] (801) {$\bullet$};
	\node[pointer,above of=801] (811) {$\bullet$};
	\node[pointer,above of=811] (821) {$\bullet$};
	\node[data,above of=800,minimum height=4.5em] (800d) {\small $21$};
	
	%State 9
	\node[data,right of=801,xshift=1em] (900) {\small $25$};
	\node[pointer,right of=900] (901) {$\bullet$};
	
	%State 10
	\node[holder,right of=901,xshift=1em] (1000) {};
	\node[holder,above of=1000] (1010) {};
	\node[holder,above of=1010] (1020) {};
	\node[data,above of=1000,minimum height=4.5em] (1000d) {$\bot$};

	%Level 0 edges
	\foreach \i/\j in {0/1,1/2,2/3,3/4,4/5,5/6,6/7,7/8,8/9,9/10}{
	  \draw[p] (\i01.center) -> (\j00);
	}
	
	%Level 1 edges
	\foreach \i/\j in {0/2,2/4,4/6,6/8,8/10}{
	  \draw[p] (\i11.center)-> (\j10);
	}
	
	%Level 2 edges
	\foreach \i/\j in {0/4,4/8,8/10}{
	  \draw[p] (\i21.center)-> (\j20);
	}
      \end{tikzpicture}
      
      \bigskip
      
      \begin{tikzpicture}[
	pointer/.style={draw,thick,minimum width=1.5em,minimum height=1.5em},
	data/.style={draw,thick,minimum width=1.5em,minimum height=1.5em},
	holder/.style={minimum width=1.5em,minimum height=1.5em},
	p/.style={->,thick},
	node distance=1.5em]
	%State 0
	\node[pointer] (001) {$\bullet$};
	\node[pointer,above of=001] (011) {$\bullet$};
	\node[pointer,above of=011] (021) {$\bullet$};
	\node[pointer,above of=021] (031) {$\bullet$};
	
	%State 1
	\node[data,right of=001,xshift=1em] (100) {$3$};
	\node[pointer,right of=100] (101) {$\bullet$};
	
	%State 000010
	\node[holder,right of=101,xshift=1em] (200) {};
	\node[holder,above of=200] (210) {};
	\node[pointer,right of=200] (201) {$\bullet$};
	\node[pointer,above of=201] (211) {$\bullet$};
	\node[data,above of=200,minimum height=3em,yshift=-0.75em] (200d) {$6$};
	
	
	%State 3
	\node[data,right of=201,xshift=1em] (300) {$7$};
	\node[pointer,right of=300] (301) {$\bullet$};
	
	%State 4
	\node[holder,right of=301,xshift=1em] (400) {};
	\node[holder,above of=400] (410) {};
	\node[holder,above of=410] (420) {};
	\node[pointer,right of=400] (401) {$\bullet$};
	\node[pointer,above of=401] (411) {$\bullet$};
	\node[pointer,above of=411] (421) {$\bullet$};
	\node[data,above of=400,minimum height=4.5em] (400d) {$9$};

	%State 5
	\node[data,right of=401,xshift=1em] (500) {\small $12$};
	\node[pointer,right of=500] (501) {$\bullet$};
	
	%State 6
	\node[holder,right of=501,xshift=1em] (600) {};
	\node[holder,above of=600] (610) {};
	\node[pointer,right of=600] (601) {$\bullet$};
	\node[pointer,above of=601] (611) {$\bullet$};
	\node[data,above of=600,minimum height=3em,yshift=-0.75em] (600d) {\small $17$};
	
	%State 7
	\node[data,right of=601,xshift=1em] (700) {\small $19$};
	\node[pointer,right of=700] (701) {$\bullet$};
	
	%State 8
	\node[holder,right of=701,xshift=1em] (800) {};
	\node[holder,above of=800] (810) {};
	\node[holder,above of=810] (820) {};
	\node[holder,above of=820] (830) {};
	\node[pointer,right of=800] (801) {$\bullet$};
	\node[pointer,above of=801] (811) {$\bullet$};
	\node[pointer,above of=811] (821) {$\bullet$};
	\node[pointer,above of=821] (831) {$\bullet$};
	\node[data,above of=800,minimum height=6em,yshift=0.75em] (800d) {\small $21$};
	
	%State 9
	\node[data,right of=801,xshift=1em] (900) {\small $25$};
	\node[pointer,right of=900] (901) {$\bullet$};
	
	%State 10
	\node[holder,right of=901,xshift=1em] (1000) {};
	\node[holder,above of=1000] (1010) {};
	\node[holder,above of=1010] (1020) {};
	\node[holder,above of=1020] (1030) {};
	\node[data,above of=1000,minimum height=6em,yshift=0.75em] (1000d) {$\bot$};

	%Level 0 edges
	\foreach \i/\j in {0/1,1/2,2/3,3/4,4/5,5/6,6/7,7/8,8/9,9/10}{
	  \draw[p] (\i01.center) -> (\j00);
	}
	
	%Level 1 edges
	\foreach \i/\j in {0/2,2/4,4/6,6/8,8/10}{
	  \draw[p] (\i11.center)-> (\j10);
	}
	
	%Level 2 edges
	\foreach \i/\j in {0/4,4/8,8/10}{
	  \draw[p] (\i21.center)-> (\j20);
	}
	
	%Level 3 edges
	\foreach \i/\j in {0/8,8/10}{
	  \draw[p] (\i31.center)-> (\j30);
	}
      \end{tikzpicture}
      \caption{Linked list vs.\ skip lists}
    \end{figure}


  \subsection{\cite{CP18} Construction}
  \subsection{The Sloth Hash Function}
    \begin{enumerate}
      \item \cite{LW17}
      \item Assumptions: computing square-root requires logarithmically many squarings
    \end{enumerate}


\section{Warm-up: PoSW from Skip Lists}
  We start with basic construction that uses an ensemble of PRPs and the skip list, and then show that it is a proof of sequential work.
  
  \subsection{Construction}
    The construction takes as input a time parameter $N=2^n$ for $n\in\NN$ and two statistical parameters $w,t\in\NN$. We assume an ensemble of random permutations $P:=\seq{P_0}{P_n}$ with $P_i:\str[(i+1)\cdot w]\to\str[(i+1)\cdot w]$ sampled uniformly at random from $\sP_{(i+1)\cdot w}$, the set of permutations on $(i+1)\cdot w$-bit-long strings in $\str$. Let $P^{-1}$ denote the ``inverse'' oracle of $P$.
    \begin{enumerate}
      \item The verifier sends the statement $\chi\in\str[(n+1)\cdot w]$ to the prover
      \item The prover computes the sequence of states $\seq{\sigma_0}{\sigma_N}$ and sets $\phi=\sigma_N$ and $\phi_p=\seq{\sigma_0}{\sigma_N}$. (We will see later how one can trade-off space for time just like in \cite{CP18}.)
      \item The verifier, on receipt of $\phi$, challengers the prover on $t$ random leaf nodes $\seq{\gamma_1}{\gamma_t}$, where $\gamma_i\in[N]$.
    \end{enumerate}
    
    \begin{definition}
      An adversary $\AA$, with oracle access to $P$ and $P^{-1}$, on an input $x\in\str[w]$ outputs a $P$-sequence $\seq{x_0}{x_s}$ of length $s$ if 
	\begin{enumerate}
	  \item $x\subset x_0$, where $\subset$ denotes that $x$ is a continuous substring of $x_0$ (\ie, $x_0$ is of the form $a\circ x \circ b$ for some $a,b\in\str$).
	  \item For all $j\in[0,s-1]$, there exists some $i\in[0,n]$ such that $P_i(x_j)\subset x_{j+1}$, where $\subset$ denotes that some continuous substring of $P_i(x_j)$ of length $w$ is present as a continuous subsequence of $x_{j+1}$.
	\end{enumerate}
    \end{definition}
    
    \begin{claim}
      The probability that an adversary outputs a $P$-sequence of length $s$ making (strictly) less than $s$ sequential queries is 
	\[2q\cdot\frac{Q+\sum_{i=0}^s \size{x_i^2}}{2^{(t-1)w}},\]
      where $q$ denotes the total number of queries that the adversary is allowed to make to the random permutations and $Q$ their total length.
    \end{claim}
    \begin{proof}[``Proof'']
      The three ways that $\AA$ can output a $P$-sequence $\seq{x_0}{x_s}$ making less than $s$ queries are given below.
	\begin{enumerate}
	  \item Lucky guess of a value of $P$: for some $i\in[s],j\in[n]$ it holds that $P(x_i)\subset x_{i+1}$ and the adversary \TE{did not} make the query $P(x_i)$. As $P$ is random, the probability of this event can be upper-bounded by 
	    \[q\cdot\frac{\sum_{i=0}^s \size{x_i^2}}{2^{(t-1)w}}?\]
	  \item Collision: The $x_j$s were not computed sequentially. That is it holds that for some $0\leq j< k\leq s-1$ a query $x_j$ is made in round $j$ and a query $x_k$ is made in round $k$ where $P(x_j)\subset x_k$. Again, since $P$ is uniformly random, the probability is 
	    \[q\cdot\frac{Q^2}{2^{(t-1)w}}?\]
	  \item $1.$ or $2.$ occurs with $P^{-1}$: similar bounds hold.
	\end{enumerate}
      The original bound follows by a union bound.
    \end{proof}

    \begin{figure}
      \begin{tikzpicture}[
	state/.style={draw,thick,minimum width=2em,minimum height=2em},
	dstate/.style={draw,dotted,minimum width=2em,minimum height=2em},
	p/.style={->,thick},
	P/.style={-,thick,dashed},
	node distance=2em]
	%State 0
	\node[state] (00) {$\chi_0$};
	\node[state,above of=00] (01) {$\chi_1$};
	\node[state,above of=01] (02) {$\chi_2$};
	\node[state,above of=02] (03) {$\chi_3$};
	
	%State 1
	\node[state,right of=00,xshift=2em] (10) {$\sigma_{1,0}$};
	\node[dstate,above of=10] (11) {};
	\node[dstate,above of=11] (12) {};
	\node[dstate,above of=12] (13) {};
	
	%State 000010
	\node[state,right of=10,xshift=2em] (20) {$\sigma_{2,0}$};
	\node[state,above of=20] (21) {$\sigma_{2,1}$};
	\node[dstate,above of=21] (22) {};
	\node[dstate,above of=22] (23) {};
	
	%State 3
	\node[state,right of=20,xshift=2em] (30) {$\sigma_{3,0}$};
	\node[dstate,above of=30] (31) {};
	\node[dstate,above of=31] (32) {};
	\node[dstate,above of=32] (33) {};
	
	%State 4
	\node[state,right of=30,xshift=2em] (40) {$\sigma_{4,0}$};
	\node[state,above of=40] (41) {$\sigma_{4,1}$};
	\node[state,above of=41] (42) {$\sigma_{4,2}$};
	\node[dstate,above of=42] (43) {};

	%State 5
	\node[state,right of=40,xshift=2em] (50) {$\sigma_{5,0}$};
	\node[dstate,above of=50] (51) {};
	\node[dstate,above of=51] (52) {};
	\node[dstate,above of=52] (53) {};
	
	%State 6
	\node[state,right of=50,xshift=2em] (60) {$\sigma_{6,0}$};
	\node[state,above of=60] (61) {$\sigma_{6,1}$};
	\node[dstate,above of=61] (62) {};
	\node[dstate,above of=62] (63) {};
	
	%State 7
	\node[state,right of=60,xshift=2em] (70) {$\sigma_{7,0}$};
	\node[dstate,above of=70] (71) {};
	\node[dstate,above of=71] (72) {};
	\node[dstate,above of=72] (73) {};
	
	%State 8
	\node[state,right of=70,xshift=2em] (80) {$\sigma_{8,0}$};
	\node[state,above of=80] (81) {$\sigma_{8,1}$};
	\node[state,above of=81] (82) {$\sigma_{8,2}$};
	\node[state,above of=82] (83) {$\sigma_{8,3}$};
	
	%Level 0 edges
	\foreach \i/\j in {0/1,1/2,2/3,3/4,4/5,5/6,6/7,7/8}{
	  \draw[p] (\i0) -> (\j0);
	}
	
	%Level 1 edges
	\foreach \i/\j in {0/2,2/4,4/6,6/8}{
	  \draw[p] (\i1)-> (\j1);
	}
	
	%Level 2 edges
	\foreach \i/\j in {0/4,4/8}{
	  \draw[p] (\i2)-> (\j2);
	}
	
	%Level 3 edges
	\foreach \i/\j in {0/8}{
	  \draw[p] (\i3)-> (\j3);
	}
	
	%Permutation P_1
	\foreach \i in {1,3,5,7}{
	  \draw[P] ([xshift=-1em]\i0.north west) -> ([xshift=-1em]\i0.south west);
	}
	
	%Permutation P_2
	\foreach \i in {2,6}{
	  \draw[P] ([xshift=-1em]\i1.north west) -> ([xshift=-1em]\i0.south west);
	}
	
	%Permutation P_3
	\draw[P] ([xshift=-1em]42.north west) -> ([xshift=-1em]40.south west);
	
	%Permutation P_4
	\draw[P] ([xshift=-1em]83.north west) -> ([xshift=-1em]80.south west);
      \end{tikzpicture}
      \caption{Schematic diagram for $N=8$. The statement is of the form $\chi=\chi_0\circ\cdots\circ\chi_3$ and is the initial state $\sigma_0$ in the construction. The proof, $\phi$, is the final state $\sigma_8=\sigma_0\circ\cdots\circ\sigma_3$. The vertical lines represent an application of the permutation $P$ with the length indicative of the exact permutation: e.g., the rightmost vertical line denotes an application of $P_4$ to the input $\chi_3\circ\sigma_{4,2}\circ\sigma_{6,1}\circ\sigma_{7,0}$.}
    \end{figure}

  
    \begin{sidewaysfigure}
    \centering
    \caption{The red and blue box represent two (independent) challenges $c_1=6$ and $c_2=13$. The red (resp., blue) path is the shortest path from $0$ to $16$ that goes through $6$ (resp., $13$). The (sub)states with red (resp., blue) bullets are required for the verification of the red (resp., blue) path. The shaded boxes are recomputed during the verification.}
    \begin{tikzpicture}[
      node distance=3em,
      state/.style={draw,thick,minimum width=1em,minimum height=3em},
      cstate/.style={draw,thick,minimum width=1em,minimum height=3em,fill=red},
      restate/.style={draw,thick,minimum width=1em,minimum height=3em,pattern=north west lines, pattern color=red},
      P/.style={-,thick,dashed},
      p/.style={->,thick}]
      %State 0
      \node[state] (00) {};
      \node[state,above of=00] (01) {};
      \node[state,above of=01] (02) {};
      \node[state,above of=02] (03) {};
      \node[state,above of=03] (04) {};
      
      %State 1
      \node[state,right of=00] (10) {};
      
      %State 00010
      \node[state,right of=10] (20) {};
      \node[state,above of=20] (21) {$\rbullet$};
      
      %State 3
      \node[state,right of=20] (30) {$\rbullet$};
      
      %State 4
      \node[restate,right of=30] (40) {};
      \node[restate,above of=40] (41) {};
      \node[restate,above of=41] (42) {};

      %State 5
      \node[state,right of=40] (50) {$\rbullet$};
      
      %State 6
      \node[cstate,right of=50] (60) {};
      \node[cstate,above of=60] (61) {};
      
      %State 7
      \node[state,right of=60] (70) {$\rbullet$};
      
      %State 8
      \node[restate,right of=70] (80) {};
      \node[restate,above of=80] (81) {};
      \node[restate,above of=81] (82) {};
      \node[restate,above of=82] (83) {};
      
      %State 9
      \node[state,right of=80] (90) {};
      
      %State 10
      \node[state,right of=90] (100) {};
      \node[state,above of=100] (101) {};
      
      %State 11
      \node[state,right of=100] (110) {};
      
      %State 12
      \node[state,right of=110] (120) {};
      \node[state,above of=120] (121) {};
      \node[state,above of=121] (122) {$\rbullet$};

      %State 13
      \node[state,right of=120] (130) {};
      
      %State 14
      \node[state,right of=130] (140) {};
      \node[state,above of=140] (141) {$\rbullet$};
      
      %State 15
      \node[state,right of=140] (150) {$\rbullet$};
      
      %State 16
      \node[state,right of=150] (160) {};
      \node[state,above of=160] (161) {};
      \node[state,above of=161] (162) {};
      \node[state,above of=162] (163) {};
      \node[state,above of=163] (164) {};
      
      %\Labels
      \foreach \i in {0,...,16}{
	\node[below of=\i0] {\i};
      }
      
      %Level 0 edges
      \foreach \i/\j in {0/1,1/2,2/3,3/4,4/5,5/6,6/7,7/8,8/9,9/10,10/11,11/12,12/13,13/14,14/15,15/16}{
	\draw[p] (\i0) -> (\j0);
      }
      
      %Level 1 edges
      \foreach \i/\j in {0/2,2/4,4/6,6/8,8/10,10/12,12/14,14/16}{
	\draw[p] (\i1)-> (\j1);
      }
      
      %Level 2 edges
      \foreach \i/\j in {0/4,4/8,8/12,12/16}{
	\draw[p] (\i2)-> (\j2);
      }
      
      %Level 3 edges
      \foreach \i/\j in {0/8,8/16}{
	\draw[p] (\i3)-> (\j3);
      }
      
      \draw[p] (04) -> (164);
      
      %Shortest path for red
      \foreach \i/\j in {02/42,41/61,61/81,83/163}{
	\draw[p,red] (\i) -> (\j);
      }
      
      %Permutation P_1
      \foreach \i in {1,3,5,7,9,11,13,15}{
	\draw[P] ([xshift=-1em]\i0.north west) -> ([xshift=-1em]\i0.south west);
      }
      
      %Permutation P_2
      \foreach \i in {2,6,10,14}{
	\draw[P] ([xshift=-1em]\i1.north west) -> ([xshift=-1em]\i0.south west);
      }
      
      %Permutation P_3
      \foreach \i in {4,8,12}{
	\draw[P] ([xshift=-1em]\i2.north west) -> ([xshift=-1em]\i0.south west);
      }
      
      %Permutation P_4
      \draw[P] ([xshift=-1em]83.north west) -> ([xshift=-1em]80.south west);
      
      %Permutation P_5
      \draw[P] ([xshift=-1em]164.north west) -> ([xshift=-1em]160.south west);
    
    \end{tikzpicture}
     \begin{tikzpicture}[
      node distance=3em,
      state/.style={draw,thick,minimum width=1em,minimum height=3em},
      cstate/.style={draw,thick,minimum width=1em,minimum height=3em,fill=blue},
      restate/.style={draw,thick,minimum width=1em,minimum height=3em,pattern=north west lines, pattern color=blue},
      P/.style={-,thick,dashed},
      p/.style={->,thick}]
      
      %State 0
      \node[state] (00) {};
      \node[state,above of=00] (01) {};
      \node[state,above of=01] (02) {};
      \node[state,above of=02] (03) {};
      \node[state,above of=03] (04) {};
      
      %State 1
      \node[state,right of=00] (10) {};
      
      %State 00010
      \node[state,right of=10] (20) {};
      \node[state,above of=20] (21) {};
      
      %State 3
      \node[state,right of=20] (30) {};
      
      %State 4
      \node[state,right of=30] (40) {};
      \node[state,above of=40] (41) {};
      \node[state,above of=41] (42) {$\bbullet$};

      %State 5
      \node[state,right of=40] (50) {};
      
      %State 6
      \node[state,right of=50] (60) {};
      \node[state,above of=60] (61) {$\bbullet$};
      
      %State 7
      \node[state,right of=60] (70) {$\bbullet$};
      
      %State 8
      \node[restate,right of=70] (80) {};
      \node[restate,above of=80] (81) {};
      \node[restate,above of=81] (82) {};
      \node[restate,above of=82] (83) {};
      
      %State 9
      \node[state,right of=80] (90) {};
      
      %State 10
      \node[state,right of=90] (100) {};
      \node[state,above of=100] (101) {$\bbullet$};
      
      %State 11
      \node[state,right of=100] (110) {$\bbullet$};
      
      %State 12
      \node[restate,right of=110] (120) {};
      \node[restate,above of=120] (121) {};
      \node[restate,above of=121] (122) {};

      %State 13
      \node[state,right of=120,fill=blue] (130) {};
      
      %State 14
      \node[restate,right of=130] (140) {};
      \node[restate,above of=140] (141) {};
      
      %State 15
      \node[state,right of=140] (150) {$\bbullet$};
      
      %State 16
      \node[state,right of=150] (160) {};
      \node[state,above of=160] (161) {};
      \node[state,above of=161] (162) {};
      \node[state,above of=162] (163) {};
      \node[state,above of=163] (164) {};
      
      %\Labels
      \foreach \i in {0,...,16}{
	\node[below of=\i0] {\i};
      }
      
      %Level 0 edges
      \foreach \i/\j in {0/1,1/2,2/3,3/4,4/5,5/6,6/7,7/8,8/9,9/10,10/11,11/12,12/13,13/14,14/15,15/16}{
	\draw[p] (\i0) -> (\j0);
      }
      
      %Level 1 edges
      \foreach \i/\j in {0/2,2/4,4/6,6/8,8/10,10/12,12/14,14/16}{
	\draw[p] (\i1)-> (\j1);
      }
      
      %Level 2 edges
      \foreach \i/\j in {0/4,4/8,8/12,12/16}{
	\draw[p] (\i2)-> (\j2);
      }
      
      %Level 3 edges
      \foreach \i/\j in {0/8,8/16}{
	\draw[p] (\i3)-> (\j3);
      }
      
      \draw[p] (04) -> (164);
      
      %Shortest path for blue
      \foreach \i/\j in {03/83,82/122,120/130,130/140,141/161}{
	\draw[p,blue] (\i) -> (\j);
      }
      
      %Permutation P_1
      \foreach \i in {1,3,5,7,9,11,13,15}{
	\draw[P] ([xshift=-1em]\i0.north west) -> ([xshift=-1em]\i0.south west);
      }
      
      %Permutation P_2
      \foreach \i in {2,6,10,14}{
	\draw[P] ([xshift=-1em]\i1.north west) -> ([xshift=-1em]\i0.south west);
      }
      
      %Permutation P_3
      \foreach \i in {4,8,12}{
	\draw[P] ([xshift=-1em]\i2.north west) -> ([xshift=-1em]\i0.south west);
      }
      
      %Permutation P_4
      \draw[P] ([xshift=-1em]83.north west) -> ([xshift=-1em]80.south west);
      
      %Permutation P_5
      \draw[P] ([xshift=-1em]164.north west) -> ([xshift=-1em]160.south west);
    \end{tikzpicture}
  \end{sidewaysfigure}
  \subsection{Trade-off}
\section{Main Construction}
  
  
\bibliographystyle{alpha}
\bibliography{Skiplist.bib}
\end{document}

  